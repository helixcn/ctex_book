\documentclass[utf8]{book}
\usepackage{titletoc}
\usepackage{titlesec}
\usepackage{ctexcap}
\usepackage[b5paper,text={125mm,195mm},centering,left=1in,right=1in,top=1in,bottom=1in]{geometry}
\usepackage[]{geometry}
\usepackage{imakeidx}
\usepackage{multicol}
\usepackage{hyperref}
\makeindex
\bibliographystyle{plain}
\begin{document}
\title{\heiti 中国植物多样性地理图集}
\author{\fangsong 中国科学院植物研究所 编著}
\date{2018年1月}

\frontmatter
\maketitle

\chapter{序 I}

这部分是序言这部分是序言这部分是序言这部分是序言这部分是序言这部分是序言这部分是序言这部分是序言这部分是序言这部分是序言这部分是序言这部分是序言这部分是序言这部分是序言这部分是序言这部分是序言这部分是序言这部分是序言这部分是序言这部分是序言这部分是序言这部分是序言这部分是序言这部分是序言这部分是序言

\chapter{序 II}

这部分是序言这部分是序言这部分是序言这部分是序言这部分是序言这部分是序言这部分是序言这部分是序言这部分是序言这部分是序言这部分是序言这部分是序言这部分是序言这部分是序言这部分是序言这部分是序言这部分是序言这部分是序言这部分是序言这部分是序言这部分是序言这部分是序言这部分是序言这部分是序言这部分是序言这部分是序言这部分是序言这部分是序言这部分是序言这部分是序言这部分是序言这部分是序言这部分是序言这部分是序言这部分是序言这部分是序言这部分是序言这部分

\chapter{前~言}

这部分是前言这部分是前言这部分是前言这部分是前言这部分是前言这部分是前言这部分是前言这部分是前言这部分是前言这部分是前言这部分是前言这部分是前言这部分是前言这部分是前言这部分是前言这部分是前言这部分是前言这部分是前言这部分是前言这部分是前言这部分是前言这部分是前言这部分是前言这部分是前言这部分是前言这部分是前言这部分是前言这部分是前言这部分是前言这部分是前言这部分是前言这部分是前言这部分是前言这部分是前言这部分是前言这部分是前言这部分是前言这部分

这部分是前言

\renewcommand\contentsname{目~录}
\tableofcontents

\mainmatter

\part{总论}

\chapter{中国植物调查的历史}

\section{一级节标题}

\subsection{二级节标题}

\subsubsection{三级节标题}

\begin{multicols}{2}

清朝中期, 外国人就开始中国采集植物标本。清朝中期, 外国人就开始中国采集植物标本。清朝中期, 外国人就开始中国采集植物标本。清朝中期, 外国人就开始中国采集植物标本。清朝中期, 外国人就开始中国采集植物标本。清朝中期, 外国人就开始中国采集植物标本。

\end{multicols}

\chapter{植物区系分区}

\section{一级节标题}

\subsection{二级节标题}

\subsubsection{三级节标题}

\begin{multicols}{2}
和标本采集植物调查和标本采集植物调查和标本采集植物调查和标本采集植物调查和标本采集植物调查和标本采集植物调查和标本采集植物调查和标本采集植物调查和标本采集植物调查和标本采集植物调查和标本采集植物调查和标本采集植物调查和标本采集植物调查和标本采集植物调查和标本采集植物调查和标本采集植物调查和标本采集植物调查和标本采集植物调查和标本采集植物调查和标本采集植物\index{调查}和标本采集植物调查和标本采集植物调查和标本采集植物调查和标本采集

\end{multicols}

\part{植物多样性分区概述}

\chapter{东北地区}

\begin{multicols}{2}
Lorem ipsum dolor sit amet, consectetur adipisicing elit, sed do eiusmod tempor incididunt ut labore et dolore magna aliqua. Ut enim ad minim veniam, quis nostrud exercitation ullamco laboris nisi ut aliquip ex ea commodo consequat. Duis aute irure dolor in \index{reprehenderit} in voluptate velit esse cillum dolore eu fugiat nulla pariatur. Excepteur sint occaecat cupidatat non proident, sunt in culpa qui officia deserunt mollit anim id est laborum.
\end{multicols}

\chapter{华北地区}
\begin{multicols}{2}
Lorem ipsum dolor sit amet, consectetur adipisicing elit, sed do eiusmod tempor incididunt ut labore et dolore magna aliqua. Ut enim ad minim veniam, quis nostrud exercitation ullamco laboris nisi ut aliquip ex ea commodo consequat. Duis aute irure dolor in reprehenderit in voluptate velit esse cillum dolore eu fugiat nulla pariatur. Excepteur sint occaecat cupidatat non proident, sunt in culpa qui officia deserunt mollit anim id est laborum.
\end{multicols}

\include{angiosperms}

\appendix

\chapter{APG分类系统的科名}

内容附录内容附录内容附录内容附录内容附录内容附录内容附录内容附录内容附录内容附录内容附录内容附录内容附录内容附录内容附录内容附录内容附录内容文内容正文内容正文内容正文\index{内容}

内容附录内容附录内容附录内容附录内容附录内容附录内容附录内容附录内容附录内容附录内容附录内容附录内容附录内容附录内容附录内容附录内容附录内容附录内容附录内容附录内容正文内容正文\cite{DK1}.

\renewcommand\indexname{索~~引}
\printindex
\addcontentsline{toc}{chapter}{索~引}

\backmatter

\addcontentsline{toc}{chapter}{参考文献}

\begin{thebibliography}{参考文献}
\bibitem[Knuth1 et al. 1997]{DK1} D. Knuth, T.A.O.C.P. , Vol. 1, Addison-Wesley, 1997.
\bibitem[Knuth2]{DK2} D. Knuth, T.A.O.C.P. , Vol. 2, Addison-Wesley, 1997.
\bibitem[TONG YH 2014]{TONG} TONG YH, PANG KS, XIAN NH, 2014. Carpinus insularis (Betulaceae), A new species from Hong Kong [J]. J Trop Subtrop Bot, 22(2): 121-124. [童毅华, 彭权森, 夏念和,2014. 香港桦木科一新种——香港鹅耳枥 [J]. 热带亚热带植物学报,22(2):121-124.]
\bibitem[XIA NH 2008]{XIA} XIA NH, DENG YF, YIP KL, 2008. Syzygium impressum (Myrtaceae), A new species from Hong Kong [J]. J Trop Subtrop Bot, 16(1):19-22. [夏念和,邓云飞,叶国梁,2008. 香港桃金娘科一新种-凹脉赤楠 [J].热带亚热带植物学报,16(1):19-22.]
\end{thebibliography}

\chapter{后~~记}

后记内容后记内容后记内容后记内容后记内容后记内容后记内容后记内容后记内容后记内容后记内容后记内容后记内容后记内容后记内容后记内容后记内容后记内容后记内容后记内容后记内容后记内容后记内容后记内容后记内容后记内容后记内容后记内容后记内容后记内容后记内容后记内容后记内容后记内容后记内容后记内容后记内容后记内容后记内容后记内容后记内容后记内容后记内容后记内容后记内容后记内容后记内容后记内容后记内容后记内容后记内容后记内容后记内容后记内容后记内容后记内容后记内容后记内容后记内容后记内容后记内容后记内容后记内容后记内容后记内容

\begin{flushright}
作~~者~~~~~~~~~

2018年1月~~~~~
\end{flushright}

\end{document}
